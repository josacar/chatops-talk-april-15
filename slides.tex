%%%%%%%%%%%%%%%%%%%%%%%%%%%%%%%%%%%%%%%%%
% KOMA-Script Presentation
% LaTeX Template
% Version 1.0 (3/3/13)
%
% This template has been downloaded from:
% http://www.LaTeXTemplates.com
%
% Original Authors:
% Marius Hofert (marius.hofert@math.ethz.ch)
% Markus Kohm (komascript@gmx.info)
% Described in the PracTeX Journal, 2010, No. 2
%
% License:
% CC BY-NC-SA 3.0 (http://creativecommons.org/licenses/by-nc-sa/3.0/)
%
%%%%%%%%%%%%%%%%%%%%%%%%%%%%%%%%%%%%%%%%%

%----------------------------------------------------------------------------------------
%saPACKAGES AND OTHER DOCUMENT CONFIGURATIONS
%----------------------------------------------------------------------------------------

\documentclass[
paper=128mm:96mm, % The same paper size as used in the beamer class
fontsize=11pt, % Font size
pagesize, % Write page size to dvi or pdf
parskip=half-, % Paragraphs separated by half a line
]{scrartcl} % KOMA script (article)

\linespread{1.12} % Increase line spacing for readability

%------------------------------------------------
% Colors
\usepackage{xcolor} % Required for custom colors
% Define a few colors for making text stand out within the presentation
\definecolor{mygreen}{RGB}{44,85,17}
\definecolor{myblue}{RGB}{34,31,217}
\definecolor{mybrown}{RGB}{194,164,113}
\definecolor{myred}{RGB}{255,66,56}
% Use these colors within the presentation by enclosing text in the commands below
\newcommand*{\mygreen}[1]{\textcolor{mygreen}{#1}}
\newcommand*{\myblue}[1]{\textcolor{myblue}{#1}}
\newcommand*{\mybrown}[1]{\textcolor{mybrown}{#1}}
\newcommand*{\myred}[1]{\textcolor{myred}{#1}}
%------------------------------------------------

%------------------------------------------------
% Margins
\usepackage[ % Page margins settings
includeheadfoot,
top=3.5mm,
bottom=3.5mm,
left=5.5mm,
right=5.5mm,
headsep=6.5mm,
footskip=8.5mm
]{geometry}
%------------------------------------------------

%------------------------------------------------
% Fonts
\usepackage[T1]{fontenc} % For correct hyphenation and T1 encoding
\usepackage{lmodern} % Default font: latin modern font
%\usepackage{fourier} % Alternative font: utopia
%\usepackage{charter} % Alternative font: low-resolution roman font
\renewcommand{\familydefault}{\sfdefault} % Sans serif - this may need to be commented to see the alternative fonts
%------------------------------------------------

%------------------------------------------------
% Various required packages
\usepackage{amsthm} % Required for theorem environments
\usepackage{bm} % Required for bold math symbols (used in the footer of the slides)
\usepackage{graphicx} % Required for including images in figures
\usepackage{tikz} % Required for colored boxes
\usepackage{booktabs} % Required for horizontal rules in tables
\usepackage{multicol} % Required for creating multiple columns in slides
\usepackage{lastpage} % For printing the total number of pages at the bottom of each slide
\usepackage[english]{babel} % Document language - required for customizing section titles
\usepackage{microtype} % Better typography
\usepackage{tocstyle} % Required for customizing the table of contents
\usepackage{minted} % Code listings
\newmintedfile[coffee]{coffee}{mathescape, linenos, numbersep=1pt, fontfamily=tt, fontsize=\tiny, frame=lines, framesep=3mm}
\newmintedfile[ruby]{ruby}{mathescape, linenos, numbersep=1pt, fontfamily=tt, fontsize=\tiny, frame=lines, framesep=3mm}
\newmintedfile[python]{python}{mathescape, linenos, numbersep=1pt, fontfamily=tt, fontsize=\tiny, frame=lines, framesep=3mm}
\newminted{ruby}{mathescape, gobble=2, linenos, numbersep=1pt, fontfamily=tt, fontsize=\tiny, frame=lines, framesep=3mm}
%------------------------------------------------

%------------------------------------------------
% Slide layout configuration
\usepackage{scrpage2} % Required for customization of the header and footer
\pagestyle{scrheadings} % Activates the pagestyle from scrpage2 for custom headers and footers
\clearscrheadfoot % Remove the default header and footer
\setkomafont{pageheadfoot}{\normalfont\color{black}\sffamily} % Font settings for the header and footer

% Sets vertical centering of slide contents with increased space between paragraphs/lists
\makeatletter
\renewcommand*{\@textbottom}{\vskip \z@ \@plus 1fil}
\newcommand*{\@texttop}{\vskip \z@ \@plus .5fil}
\addtolength{\parskip}{\z@\@plus .25fil}
\makeatother

% Remove page numbers and the dots leading to them from the outline slide
\makeatletter
\newtocstyle[noonewithdot]{nodotnopagenumber}{\settocfeature{pagenumberbox}{\@gobble}}
\makeatother
\usetocstyle{nodotnopagenumber}

\AtBeginDocument{\renewcaptionname{english}{\contentsname}{\Large Outline}} % Change the name of the table of contents
%------------------------------------------------

%------------------------------------------------
% Header configuration - if you don't want a header remove this block
\ihead{
\hspace{-2mm}
\begin{tikzpicture}[remember picture,overlay]
\node [xshift=\paperwidth/2,yshift=-\headheight] (mybar) at (current page.north west)[rectangle,fill,inner sep=0pt,minimum width=\paperwidth,minimum height=2\headheight,top color=mygreen!64,bottom color=mygreen]{}; % Colored bar
\node[below of=mybar,yshift=3.3mm,rectangle,shade,inner sep=0pt,minimum width=128mm,minimum height =1.5mm,top color=black!50,bottom color=white]{}; % Shadow under the colored bar
shadow
\end{tikzpicture}
\color{white}\runninghead} % Header text defined by the \runninghead command below and colored white for contrast
%------------------------------------------------

%------------------------------------------------
% Footer configuration
% \newlength{\footheight}
\setlength{\footheight}{8mm} % Height of the footer
\addtokomafont{pagefoot}{\footnotesize} % Small font size for the footnote

\ifoot{% Left side
\hspace{-2mm}
\begin{tikzpicture}[remember picture,overlay]
\node [xshift=\paperwidth/2,yshift=\footheight] at (current page.south west)[rectangle,fill,inner sep=0pt,minimum width=\paperwidth,minimum height=3pt,top color=mygreen,bottom color=mygreen]{}; % Green bar
\end{tikzpicture}
\myauthor\ \raisebox{0.2mm}{$\bm{\vert}$}\ \myuni % Left side text
}

\ofoot[\pagemark/\pageref{LastPage}\hspace{-2mm}]{\pagemark/\pageref{LastPage}\hspace{-2mm}} % Right side
%------------------------------------------------

%------------------------------------------------
% Section spacing - deeper section titles are given less space due to lesser importance
\usepackage{titlesec} % Required for customizing section spacing
\titlespacing{\section}{0mm}{0mm}{0mm} % Lengths are: left, before, after
\titlespacing{\subsection}{0mm}{0mm}{-1mm} % Lengths are: left, before, after
\titlespacing{\subsubsection}{0mm}{0mm}{-2mm} % Lengths are: left, before, after
\setcounter{secnumdepth}{0} % How deep sections are numbered, set to no numbering by default - change to 1 for numbering sections, 2 for numbering sections and subsections, etc
%------------------------------------------------

%------------------------------------------------
% Theorem style
\newtheoremstyle{mythmstyle} % Defines a new theorem style used in this template
{0.5em} % Space above
{0.5em} % Space below
{} % Body font
{} % Indent amount
{\sffamily\bfseries} % Head font
{} % Punctuation after head
{\newline} % Space after head
{\thmname{#1}\ \thmnote{(#3)}} % Head spec

\theoremstyle{mythmstyle} % Change the default style of the theorem to the one defined above
\newtheorem{theorem}{Theorem}[section] % Label for theorems
\newtheorem{remark}[theorem]{Remark} % Label for remarks
\newtheorem{algorithm}[theorem]{Algorithm} % Label for algorithms
\makeatletter % Correct qed adjustment
%------------------------------------------------

%------------------------------------------------
% The code for the box which can be used to highlight an element of a slide (such as a theorem)
\newcommand*{\mybox}[2]{ % The box takes two arguments: width and content
\par\noindent
\begin{tikzpicture}[mynodestyle/.style={rectangle,draw=mygreen,thick,inner sep=2mm,text justified,top color=white,bottom color=white,above}]\node[mynodestyle,at={(0.5*#1+2mm+0.4pt,0)}]{ % Box formatting
\begin{minipage}[t]{#1}
#2
\end{minipage}
};
\end{tikzpicture}
\par\vspace{-1.3em}}
%------------------------------------------------

%----------------------------------------------------------------------------------------
%3emPRESENTATION INFORMATION
%----------------------------------------------------------------------------------------

\newcommand*{\mytitle}{ChatOps: Operations meet chat} % Title
\newcommand*{\runninghead}{ChatOps: Operations meet chat} % Running head displayed on almost all slides
\newcommand*{\myauthor}{Jose Luis Salas} % Presenters name(s)
\newcommand*{\mydate}{\today} % Presentation date
\newcommand*{\myuni}{Valencia DevOps -- April 2015} % University or department

%----------------------------------------------------------------------------------------

\begin{document}

%----------------------------------------------------------------------------------------
%documentTITLE SLIDE
%----------------------------------------------------------------------------------------

% Title slide - you may have to tweak a few of the numbers if you wish to make changes to the layout
\thispagestyle{empty} % No slide header and footer
\begin{tikzpicture}[remember picture,overlay] % Background box
\node [xshift=\paperwidth/2,yshift=\paperheight/2] at (current page.south west)[rectangle,fill,inner sep=0pt,minimum width=\paperwidth,minimum height=\paperheight/3,top color=mygreen,bottom color=mygreen]{}; % Change the height of the box, its colors and position on the page here
\end{tikzpicture}
% Text within the box
\begin{flushright}
\vspace{0.6cm}
\color{white}\sffamily
{\bfseries\Large\mytitle\par} % Title
\vspace{0.5cm}
\normalsize
\myauthor\par % Author name
\mydate\par % Date
\vfill
\end{flushright}

\clearpage

%----------------------------------------------------------------------------------------
%clearpageTABLE OF CONTENTS
%----------------------------------------------------------------------------------------

\thispagestyle{empty} % No slide header and footer

\small\tableofcontents % Change the font size and print the table of contents - it may be useful to shrink the font size further if the presentation is full of sections
% To exclude sections/subsections from the table of contents, put an asterisk after \(sub)section like so: \section*{Section Name}

\clearpage

%----------------------------------------------------------------------------------------
%clearpagePRESENTATION SLIDES
%----------------------------------------------------------------------------------------

\section{whoami}

I'm Jose Luis Salas

Work as SRE at peerTransfer, programming in Ruby and Shell mostly

Open source contributor: josacar/chef-babushka, josacar/lita-campfire, josacar/lita-jobs, fog/fog, sethvargo/chefspec, rumblelabs/asset\_sync, aetrion/chef-dnsimple, jesseadams/sphinx-cookbook, opscode-cookbooks/client-rekey, librato/papertrail-cookbook, sstephenson/ruby-build, hw-cookbooks/route53, josephruscio/twke, geokit/geokit, collectiveidea/tinder, berkshelf/berkshelf, librato/collectd-librato-cookbook, chef/ohai, opscode-cookbooks/rsyslog, peter-murach/github

Twitter: @josacar

GitHub: josacar

\clearpage

%------------------------------------------------

\section{What is ChatOps}

Doing Operations in a Chat tool

\subsection{What does ChatOps provide?}

\begin{enumerate}
\item Everyone sees all of that happen since their first day
\item Place tools in the middle of the conversation
\item Teaching by doing by making things visible
\item Communicate by doing
\end{enumerate}

\clearpage

%------------------------------------------------

\subsection{Questions I haven't asked since ChatOps}
\begin{itemize}
\item How is the deployment doing?
\item Should I deploy or are you deploying that?
\item Can I deploy to this environment?
\item Is the smoke suite green?
\item Is anyone checking the sensu alert?
\item Has CI passed in that branch?
\item Is the system load ok?
\item Is anyone using this environment?
\end{itemize}

\clearpage

%------------------------------------------------

\section{ChatOps in practice}
\subsection{What do you need?}

\begin{itemize}
  \item A chat client: IRC, HipChat, Slack, Flowdock, and Campfire
  \item Chat bots ( all FLOSS ):
  \begin{itemize}
    \item Hubot
    \item Lita
    \item Err
  \end{itemize}
\end{itemize}
\clearpage

%------------------------------------------------

\subsection{Hubot}
\begin{itemize}
\item Written in CoffeeScript and JavaScript
\item Runs on top of NodeJS
\item Created and used by GitHub
\item Supports multiple chats: Campfire, Slack, HipChat
\item Storage ( brain ) : Redis, PostgreSQL, MySQL
\item Supports chat and HTTP routes
\end{itemize}
\clearpage

%------------------------------------------------

pugme.coffee
\coffee{pugme.coffee}
\clearpage

%------------------------------------------------

\subsection{Lita}
\begin{itemize}
\item Written in Ruby ( requires Ruby >= 2.0 )
\item Runs on top of MRI / JRuby / Rubinius
\item Created by JimmyCuadra
\item Supports multiple chats: Campfire, Slack, HipChat, IRC
\item Storage: Redis
\item Supports chat and HTTP routes
\end{itemize}
\clearpage

%------------------------------------------------

\begin{rubycode}
  route(/^echo\s+(.+)/, :echo, command: true, restrict_to: [:testers, :committers], help: {
    'echo TEXT' => 'Replies back with TEXT.'
  })
\end{rubycode}
\begin{rubycode}
  http.get "/builds/:id", :build_info

  def build_info(request, response)
    id = request.env["router.params"][:id]
    build = MyBuildSystem.find(id)
    response.headers["Content-Type"] = "application/json"
    response.write(MultiJson.dump(build))
  end
\end{rubycode}
\begin{rubycode}
  on :connected, :greet
  def greet(payload)
    target = Source.new(room: payload[:room])
    robot.send_message(target, "Hello #{payload[:room]}!")
  end
\end{rubycode}
\begin{rubycode}
  module Lita
    module Handlers
      class HandlerWithConfig < Handler
        config :api_key
        route(/call api/, command: true) do |response|
          response.reply(ThirdPartyAPI.new(config.api_key).call)
        end
      end
      Lita.register_handler(HandlerWithConfig)
    end
  end
\end{rubycode}
\clearpage

%------------------------------------------------

pugme.rb
\ruby{pugme.rb}
\clearpage

%------------------------------------------------

\subsection{Err}
\begin{itemize}
\item Written in Python
\item Runs on top of Python 2.7 and 3.2+
\item Created by Guillaume Binet
\item Supports multiple chats: Campfire, HipChat, IRC
\item Storage ( brain ) : File
\item Supports chat and HTTP routes
\end{itemize}
\clearpage

%------------------------------------------------

hello.py
\python{hello.py}
\clearpage

%------------------------------------------------

\section{Examples}
\subsection{Nagios}
\begin{itemize}
  \item bot nagios ack database01/mysqld
  \item bot nagios check database01/mysqld
  \item bot nagios status database01/mysqld
  \item bot nagios downtime database01/mysqld
  \item bot nagios mute database01/mysqld
  \item bot nagios unmute database01/mysqld
\end{itemize}
\clearpage

%------------------------------------------------

\subsection{Deploying}
\begin{itemize}
  \item bot deploy myapp to prod/front01
  \item bot log me smoke front01
  \item bot status yellow Issue deploying.
  \item bot lbctl disable front01
  \item bot deploy myapp to prod/front01
  \item bot log me smoke front01
  \item bot status green Everything is ok.
  \item bot lbctl enable front01
\end{itemize}
\clearpage

%------------------------------------------------

\subsection{More examples}
\begin{itemize}
\item bot pingdom checks
\item bot who is on call
\item bot conns loadbalancer
\item bot graph me -10min @collectd.load(rabbitmq*)
\item bot procs unicorn
\item bot whois 1.2.3.4
\item bot ci status rails/rails
\item bot log me hooks front01
\end{itemize}

\clearpage

%------------------------------------------------

% How to include a theorem in this presentation:
% \begin{verbatim}
% \mybox{0.8\textwidth}{
% \begin{theorem}[Murphy (1949)]
% Anything that can go wrong, will go wrong.
% \end{theorem}
% }
% \end{verbatim}

% \clearpage

%------------------------------------------------


% \section{Displaying Information}

% \clearpage

%------------------------------------------------

% \subsection{Table}

% \begin{table}[h]
% \centering
% \begin{tabular}{l l l}
% \toprule
% \textbf{Treatments} & \textbf{Response 1} & \textbf{Response 2}\\
% \midrule
% Treatment 1 & 0.0003262 & 0.562 \\
% Treatment 2 & 0.0015681 & 0.910 \\
% Treatment 3 & 0.0009271 & 0.296 \\
% \bottomrule
% \end{tabular}
% \caption{Table caption}
% \end{table}

% \clearpage

%------------------------------------------------

\subsection{ChatOps @ peerTransfer}

\begin{figure}[h]
\centering\includegraphics[width=0.8\linewidth]{cibuild.png}
\end{figure}

\clearpage

%------------------------------------------------

\begin{figure}[h]
\centering\includegraphics[width=0.8\linewidth]{librato.png}
\end{figure}

\clearpage

%------------------------------------------------

\begin{figure}[h]
\centering\includegraphics[width=0.8\linewidth]{pagerduty.png}
\end{figure}

\clearpage

%------------------------------------------------

\begin{figure}[h]
\centering\includegraphics[width=1.0\linewidth]{papertrail.png}
\end{figure}

\clearpage

%------------------------------------------------

\begin{figure}[h]
\centering\includegraphics[width=0.9\linewidth]{deploy.png}
\end{figure}

\clearpage

%------------------------------------------------

\begin{figure}[h]
\centering\includegraphics[width=0.8\linewidth]{deploy_mono.png}
\end{figure}

\clearpage

%------------------------------------------------

% \subsection{Theorem}

% The most common definition of \mygreen{Murphy's Law} is as follows.

% \mybox{0.8\textwidth}{ % Example of encapsulating text in a colored box
% \begin{theorem}[Murphy (1949)]
% Anything that can go wrong, will go wrong.
% \end{theorem}
% }

% \begin{proof}
% A special case of this theorem is proven in the textbook.
% \end{proof}

% \begin{remark}
% This is a remark.
% \end{remark}

% \begin{algorithm}
% This is an algorithm.
% \end{algorithm}

% \clearpage

%------------------------------------------------

% \section{Citations}

% An example of the \texttt{\textbackslash cite} command to cite within the presentation:

% This statement requires citation \cite{Smith:2012qr}.

% \clearpage

%------------------------------------------------

\thispagestyle{empty} % No slide header and footer

\section{Bibliography}

\bibliographystyle{unsrt}
http://www.pagerduty.com/blog/what-is-chatops/

http://blog.librato.com/posts/fire-in-the-hole

http://blog.librato.com/posts/confessions-of-a-chatbot

http://slides.com/carolnichols/chatops

http://tech.toptable.co.uk/blog/2013/11/22/beginning-a-journey-to-chatops-with-hubot/

http://docs.lita.io/plugin-authoring/handlers/

\bibliography{sample}
\clearpage

%------------------------------------------------

\thispagestyle{empty} % No slide header and footer

\begin{tikzpicture}[remember picture,overlay] % Background box
\node [xshift=\paperwidth/2,yshift=\paperheight/2] at (current page.south west)[rectangle,fill,inner sep=0pt,minimum width=\paperwidth,minimum height=\paperheight/3,top color=mygreen,bottom color=mygreen]{}; % Change the height of the box, its colors and position on the page here
\end{tikzpicture}
% Text within the box
\begin{flushright}
\vspace{0.6cm}
\color{white}\sffamily
{\bfseries\LARGE Questions?\par} % Request for questions text
\vfill
\end{flushright}

\clearpage

%----------------------------------------------------------------------------------------

\thispagestyle{empty} % No slide header and footer

\begin{tikzpicture}[remember picture,overlay] % Background box
\node [xshift=\paperwidth/2,yshift=\paperheight/2] at (current page.south west)[rectangle,fill,inner sep=0pt,minimum width=\paperwidth,minimum height=\paperheight/2,top color=mygreen,bottom color=mygreen]{}; % Change the height of the box, its colors and position on the page here
\end{tikzpicture}
% Text within the box
\begin{flushright}
\vspace{0.6cm}
\color{white}\sffamily
{\bfseries\LARGE Thanks!\par}
{\bfseries\LARGE Valencia DevOps\par}
{\bfseries\LARGE April 2015\par}
\vfill
\end{flushright}

\end{document}
